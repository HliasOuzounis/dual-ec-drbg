\section{Conclusion}

In this report, we examined the need for random number generators and looked at the theory behind one of the most infamous ones, the Dual EC DRBG algorithm. Dual EC DRBG was initially designed to provide high level security to the randomly outputted values. However, it came under huge scrutiny once a potential backdoor was first theorized. The backdoor would allow attackers to find the seed of the algorithm and predict future outputs. Something that is considered catastrophic for a random number generator. As shown in this report, the backdoor was very efficient and despite some suggestions from the community to nullify it, no changes were made and it was endorsed by NIST as a standard. Snowden's leaks in 2014 solidified the theory of the backdoor and the algorithm was finally removed from the NIST standards. Nonetheless, the algorithm is still used in many systems and some companies have yet to transition to a more secure alternative.
\\

The reputation of NIST and the NSA were tarnished from this incident and the trust in the security of the systems they endorse was shaken. The community has since been more cautious and has been more critical of the algorithms that are proposed. The Dual EC DRBG algorithm is a prime example of how a backdoor can be inserted into a system and one is left to wonder how many other algorithms have similar vulnerabilities. The need for transparency and open source code is more important than ever and the governing bodies should be more communicative of their decisions and the reasons behind them. Any suspicious attempts have to be questioned by the community to ensure a secure and private society.